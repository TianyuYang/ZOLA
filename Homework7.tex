\documentclass{article}
%\usepackage{amsmath}

% pre-amble
\title{Math 480 Final Project}
\author{Tianyu Yang}
\date{May 15, 2013}

\begin{document}



\maketitle  This is a proposal for the final project

\section{Overview}

We all use numbers and data everyday, even sometimes we did not realize.
Numbers and data have big in
uence to our daily life. For example, we can
use numbers and data to analyze what is the best time and the best way for
people traval on the highway, what is the best location for people to choose for
business, and predict how the price of one type of product goint to be and so
on.\ref{notation}.

\section{introduction}
My project will use an input data and calculate and analyze based on the data
we have. After calculation, the project will show the feedbacks such as the
distributions, statistical information, and some analysis of data. There are two
examples in my project to show how this project work

\subsection{Two Examples}\label{example}
\begin{itemize}   % {enumerate} for numbers
   \item Example 1:\\
   In this example, I will use the data of the accidents that happened on
highway I5 as my basic data. Based on the data, the Project will graph
the histgram distribution of number of injurys, number of accidents, and
number of cars. The mean, variance, confidence interval, and prodicted
time will be also shown under graph. Thus, by anaylzing the data and
distribution, we can prodict that when and where the accidents will most
likely to happen, and what is the reason for it.

   \item Example 2:\\
   In this example, assuming people want to start a new business of gas
station in Sea-Tec area. Based on the research of how the gas stations
distribute in this area and how the people lives and works in this area,
the project will give you a suggestion of reasonable place to locate the gas
station. Also by analyzing the prices and local income. The project also
give you a suggested price.
\end{itemize}
\subsection{Notation}\label{notation}
   \begin{enumerate}
     \item Formula for

$$
   y = A\alpha + B\beta + C\gamma + D\sigma

   \alpha \rightarrow \{\text{ Numbers of Gas Stations around }\}

   \beta  \rightarrow \{\text{ Numbers of people lives around }\}

   \gamma \rightarrow \{\text{ Numbers of cars passing by per day } \}

   \sigma \rightarrow \{\text{ Rents of the place per month } \}
$$

The letter A, B, C, D, are the precentage of the weight of each factors, in order to make the formula more reasonable

     \item Price of the Gas\\
     I am going to collect the data about each company's gas prices. Let the sample
data size of companys' price is equal N, and let the mean of the price is $\bar{x}$. Based
on the data we collect, we can easily calculate the Standard Diviation of the
sample data, so we let the Standard Diviation is equal $\sigma$.\\
Price Confidence Interval
when $\alpha$ = 0.05:

$$
   \{ \bar{x} - 1.96 * \frac{\sigma}{\sqrt{N}} , \bar{x} + 1.96 * \frac{\sigma}{\sqrt{N}} \}

   \bar{x} = \frac{\text{ Total prices }}{N}\\
   
   \sigma = \sqrt{\frac{1}{N} * \sum_{i=1}^{N}(x_i- \mu)^2}
$$

Thus, any price that will in the Price Confidence Interval is a reasonable price,
for the start up gas station, it is better to make the price is slightly lower than
the $\bar{x}$
   \end{enumerate}
\section{Aim or Expecting results}
In the project, I am hoping I can use the datas and my coding to calculate some statistical numbers and graph the distributions.
So I can use the numbers and distributions to actually find the best location for people to start a new gas stations business.


\end{document}
